\documentclass[../../main/thesis_msc.tex]{subfiles}


%\setcounter{chapter}{0}






\begin{document}

    \chapter{Methods}
    
    	For the puproses of this thesis we performed numerical calculations using the one dimensional, stellar evolution code \textbf{M}odules for \textbf{E}xperiments in \textbf{S}tellar \textbf{A}strophysics (\texttt{MESA}) \citep{Paxton2011, Paxton2013, Paxton2015, Paxton2018}. In this section, we present the very basic aspects of \texttt{MESA} and the physical assumptions we used in our attempt to model the evolution of single and binary stellar systems.
    	
    		\section{Modules for Experiments in Stellar Astrophysics}
    		
    			%Write 2-3 pages of the MESA basics and how it works (Newton iterations etc). Consider possible subsections
    			
    			
    		\section{Physical assumptions}
    		
    			All helium stars were created with \texttt{MESA} (version - r10398) in single star evolution by starting with a non-rotating ZAMS star of $30$ M$_{\odot}$ and solar metallicity ($Z = 0.02$). The star was left to evolve until hydrogen depletion in its centre, at which point the hydrogen envelope was artificially removed exposing a naked helium core of $\sim 9$ M$_{\odot}$ with homogeneous composition. Lower mass helium stars were created by artificially removing the outer layers of the aforementioned helium core.
    			
    				\subsection{Single stars}
    				
    					For single helium stars
    					
    				\subsection{Binary systems}
    				
    					For the binary systems
    
    
    
    
\end{document}
