\documentclass[../../main/thesis_msc.tex]{subfiles}



\begin{document}

    \chapter{Introduction}
    
		The cycle of life and death of stars  baffled astronomers for many years. The study of stellar structure and evolution continues -up to this date- to be of paramount importance, since it is crucial to our understanding of various branches of astronomy, e.g. the structure of galaxies, and chemical history of the Universe.
		
		The aim of this thesis is to investigate and get an insight in one of the most debated topics in Stellar Astrophysics; the connection between the progenitor and remnant masses, especially in the case of double neutron star (DNS) binary systems. The existence of those systems was recently confirmed by the detection of gravitational waves emitted during a merging event, accompanied by the detection of a kilonova -the "afterglow"of such an event- as described in the seminal paper of the LIGO/VIRGO collaborations \citep{ligo}.
		
		In this chapter, a synopsis that extends from the formation to the death of Helium stars will be attempted. A detailed coverage of the principles of stellar evolution is beyond the scope of this thesis and a fundamental knowledge is assumed. Moreover, for the interested reader, there are classical textbooks \citep{Kipp_book, Clayton, Prialnik, Eggleton_book} covering almost every aspect in the field of stellar astrophysics. Nevertheless, for the sake of completeness, a small introduction to several fundamental notions, tailored to our needs, will also be carried out in the next few pages. 
		

    
    
    \section{Helium stars}
    	
    	%A brief explanation of what a Helium star is
    	From the large primordial molecular clouds, protostars are being constantly formed via a process called \emph{gravoturbulent cloud fragmentation}. When the accretion of the surrounding material from the protostellar core ceases, the protostar is said to be in the \emph{pre-main sequence} (PMS) phase of its evolution, and continues to contract under the force of gravity until the central temperature becomes sufficiently high for nuclear fusion reactions on Hydrogen to occur. At this point, the star enters the \emph{main sequence} (MS) evolutionary phase as a zero-age main sequence (ZAMS) star where it will spend most of its life.
    	
    	During the MS stage, the star converts Hydrogen to Helium either via the pp-chain reactions, or via the CNO cycles, depending on its initial mass and chemical composition. Slowly but steadily, the Hydrogen in the core is being consumed by the aforementioned nuclear networks, and Helium builds up forming a Helium core. This process continues until the Hydrogen in the stellar core is depleted, resulting to an inert Hydrogen envelope engulfing the newly formed He-core; subsequently, the star exits the MS phase and the nuclear reactions in its interior that provided the necessary presure support against gravity, effectively stop. Since the star is not in an equilibrium state anymore, it starts to contract until Hydrogen is ignited in a shell around the inert Helium core. At this point, the star enters the so-called \emph{red-giant branch} (RGB) and the Hydrogen-rich envelope, on top of the H-burning shell, inflates rapidly whilst the He-core continues to contract due to the \emph{mirror principle} \citep[see][p.~369]{Kipp_book}.
    	
    	As we will explain in a moment, the Hydrogen envelope can be lost when the star is in the RGB phase, with more than one ways, exposing the He-core of the star. This naked, Hydrogen deficient, He-core is what we refer to as a \emph{Helium star}. We can classify He-stars into two groups: low-mass \emph{hot subdwarfs} (sd) that can be further subdivided into several categories (e.g. sdB, sdO) based on their spectra, and more massive \emph{Wolf-Rayet} (WR) stars that can also be subdivided into several classes (e.g. WN, WC). For a more detailed discussion we refer the reader to the work of \cite{Han2002, Han2003, Heber2009, Chiosi86, langer12}.


			\subsection{Formation of Helium stars}
			
				%A small section explaining how helium stars are being formed
				Helium stars can be formed either in isolation or as part of a binary system. In both scenarios, the physical mechanism that is responsible for the stripping of the Hydrogen envelope is of the utmost importance.
				
				In the former case of a single He-star, the necessary mass loss is being achieved due to strong, radiation-driven, stellar winds. However, the specifics of such a process have not been fully resolved yet, and an enhanced mass loss scheme, e.g. caused by rotational mixing, magnetic fields, or even strong He-flashes should be considered for the progenitor of the He-star \citep{Sweigart, Heber2009}. 
				
				In the case where the He-star progenitor is part of a binary system, the required strong mass loss can be achieved via different channels, depending on how wide the binary system is. These channels include the stable Roche-lobe overflow (RLO) and the Common Envelope (CE) ejection. We will discuss these mass loss mechanisms below.
				It should be mentioned that sdB stars can also originate from the merging of two Helium white dwarfs (He-WD) in a close binary, resulting to an object with enough mass to ignite Helium \citep{Han2002}.
				
				
			
			\subsection{Evolution of single Helium stars}
			
				%A small section explaining the evolution of single helium stars 
				Once the He-star progenitor has been stripped from its Hydrogen envelope during its RGB phase, the compression of the core continues until it reaches the necessary conditions for Helium to ignite at its centre. The ignition of core Helium burning signifies the transition to the Helium main sequence (He-MS) as a He-ZAMS star (see \hyperref[fig:T_rho_plane_ch1]{Fig 1.1}). The last two concepts are defined in a similar way to the (Hydrogen) main sequence and ZAMS respectively.
				
				During the He-MS stage, the star burns its $^4$He supplies via the triple-alpha process producing Carbon ($^{12}$C) and the stable Oxygen isotope $^{16}$O, as a byproduct. When the Helium in the core is depleted, the contraction/expansion process we described above is repeated; the idle metal core that has been formed, consists mainly by Carbon, Oxygen, Neon, and Magnisium and it is surrounded by a He-rich envelope. This whole structure will contract until Helium is ignited in a shell at the bottom of the envelope, followed by the ignition of Carbon in the centre (given that the star is massive enough). The fate of the He-star at this point depends on its mass; if it has not retain enough mass for Carbon ignition, it will gradually cool off and end its life as a Carbon/Oxygen white dwarf (CO WD). On the other hand, if it is massive enough to ignite Carbon, either on or off centre, its fate could be an Oxygen/Neon/Magnisium white dwarf (ONeMg WD), a hybrid white dwarf (CONeMg WD), or even collapse as a supernova.
				
				\begin{figure}[t]
					\centering
					\includegraphics[width=\textwidth]{../../figures/T_rho_plane_ch1.png}
					\caption{The evolution of a $3.0 M_{\odot}$ He-star, with and without mass loss. The point where the star enters the He-MS as a He-ZAMS can be seen as a small hook inside the black circle. An analytical explanation of the letters along the evolutionary tracks, is provided in the text.}
					\label{fig:T_rho_plane_ch1}
				\end{figure}
				
				To demonstrate the aforementioned stages, the evolutionary track of a $3.0 M_{\odot}$ He-star (with and without mass loss) is illustrated in the $T_c - \rho_c$ plane (\hyperref[fig:T_rho_plane_ch1]{Fig 1.1}). The letters denote the beginning and end of several phases up to the off-centre neon ignition. The A-B phase shows the contraction that follows after the RGB stage of the progenitor. The moment of core He-ignition is denoted by the black circle at point B, and signifies the entrance to the He-MS phase (B-C). At point C, Helium has been exhausted and the core contracts whilst He-shell burning follows. During the D-E phase, the Carbon is ignited in the core making a loop in the diagram around $\rho_c = 10^6$ g cm$^{-3}$. Finally, the E-F/F$^\prime$ shows the Carbon-shell burning along the contraction of the O-Ne-Mg core; at the endpoint (F/F$^\prime$), Neon is ignited off-centre.
				
				For a more complete and comprehensive overview of the evolution of single He-stars, we provide, in the form of a Kippenhahn diagram, the net energy production rate with respect to the internal structure of the $3.0 M_{\odot}$ star (with mass loss) we used in the example above (\hyperref[fig:Kipp_3p0_ch1]{Fig 1.2}).
				
				\begin{figure}[h]
					\centering
					\includegraphics[width=\textwidth]{../../figures/Kippenhahn_3p0_ch1_with_text.png}
					\caption{Kippenhahn diagramm for a $3.0 M_{\odot}$ He-star with mass loss.}
					\label{fig:Kipp_3p0_ch1}
				\end{figure}
				
				
				
				
				
				
				
					\subsubsection{Mixing mechanisms}
					
						convection, overshooting, thermohaline
						
					\subsubsection{Effects of rotation}
					
						Rotational mixing 
						
						
					\subsubsection{Transportation of angular momentum}
					
						Eddington-Sweet circulation etc
						
					\subsubsection{Winds and mass loss}
					
						Importance of mass loss in the evolution of stellar winds and Wolf-Rayet stars + magnetic braking --> connection to angular momentum losses.
						
					
				
	\section{Evolution of binary systems}
	
		Few words about how most stars form in binary systems, detached, semi-detached and contact binaries
		
			\subsection{Interaction and orbital parameters}
			
				Cases A/B/C etc
				
			\subsection{Mass transfer}
			
				Few words about mass transfer in binary systems (wind mass accretion + Roche lobe overflow)
				
			\subsection{Common envelope}
			
				Explain a little bit in more detail the basics of CE
				
			\subsection{Angular momentum transfer}
			
				Effects of angular momentum transfer + magnetic braking
				
			\subsection{Gravitational waves}
			
				The very basics for GWs and their impact on binary mergers
				
	\section{Stellar transients}
	
		Couple of words for the different types of stellar transients and how can we observe them
		
			\subsection{Classification of Supernovae}
			
				Explain in details the difference between core collapse SNe and type Ia and different subdivision
				
			\subsection{Type Ib/c Supernovae}
			
				Explain in details this particular branch
				
			\subsection{X-ray binaries}
			
				HMXB, LMXB, UCXB
    
    
    \end{document}
    