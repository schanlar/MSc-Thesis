\documentclass[../../main/thesis_msc.tex]{subfiles}



\begin{document}

    \chapter{Introduction}
    
		The cycle of life and death of stars  baffled astronomers for many years. The study of stellar structure and evolution continue to be of paramount importance up to this date, since it is crucial to our understanding of various branches of astronomy, e.g. the structure of galaxies, and chemical history of the Universe.
		
		A detailed coverage of the principles of stellar evolution is beyond the scope of this thesis. Moreover, for the interested reader, there are excellent classical textbooks \citep{Kipp_book, Clayton} covering almost every aspect in the field of stellar astrophysics. Nevertheless, for the sake of completeness, a small introduction to several fundamental notions, tailored to our needs, will be attempted in the next few paragraphs. 
		
    
    
    \section{Helium stars}
    	
    	%A brief explanation of what a Helium star is
    	From the large primordial molecular clouds, protostars are being constantly formed via a process called \emph{gravoturbulent cloud fragmentation}. When the accretion of the surrounding material from the protostellar core ceases, the protostar is said to be in the \emph{pre-main sequence} (PMS) phase of its evolution, and continues to contract under the force of gravity until the central temperature becomes sufficiently high for nuclear fusion reactions on Hydrogen to occur. At this point, the star enters the main sequence (MS) evolutionary phase as a zero-age main sequnce (ZAMS) star where it will spend most of its life.
    	
    	During the MS stage, the star converts Hydrogen to Helium either via the pp-chain reactions, or via the CNO cycles, depending on its initial mass and chemical composition. Slowly but steadily, the Hydrogen in the core is being consumed by the aforementioned nuclear networks, and Helium builds up forming a Helium core. This process continues until the Hydrogen in the stellar core is depleted, resulting to an inert Hydrogen envelope engulfing the newly formed He-core; the star exits the MS phase and begins to contract due to lack of pressure support provided by the nuclear reactions in its interior.
    	
    	As we will explain in a moment, the Hydrogen envelope can be lost with more than one ways, exposing the He-core of the star. This naked He-core is what we refer to as a \emph{Helium star}.


			\subsection{Formation of Helium stars}
			
				A small section explaining how helium stars are being formed
				
			
			\subsection{Evolution of single Helium stars}
			
				A small section explaining the evolution of single helium stars    	
				
					\subsubsection{Mixing mechanisms}
					
						convection, overshooting, thermohaline
						
					\subsubsection{Effects of rotation}
					
						Rotational mixing 
						
						
					\subsubsection{Transportation of angular momentum}
					
						Eddington-Sweet circulation etc
						
					\subsubsection{Winds and mass loss}
					
						Importance of mass loss in the evolution of stellar winds and Wolf-Rayet stars + magnetic braking --> connection to angular momentum losses.
						
					
				
	\section{Evolution of binary systems}
	
		Few words about how most stars form in binary systems, detached, semi-detached and contact binaries
		
			\subsection{Interaction and orbital parameters}
			
				Cases A/B/C etc
				
			\subsection{Mass transfer}
			
				Few words about mass transfer in binary systems (wind mass accretion + Roche lobe overflow)
				
			\subsection{Common envelope}
			
				Explain a little bit in more detail the basics of CE
				
			\subsection{Angular momentum transfer}
			
				Effects of angular momentum transfer + magnetic braking
				
			\subsection{Gravitational waves}
			
				The very basics for GWs and their impact on binary mergers
				
	\section{Stellar transients}
	
		Couple of words for the different types of stellar transients and how can we observe them
		
			\subsection{Classification of Supernovae}
			
				Explain in details the difference between core collapse SNe and type Ia and different subdivision
				
			\subsection{Type Ib/c Supernovae}
			
				Explain in details this particular branch
				
			\subsection{X-ray binaries}
			
				HMXB, LMXB, UCXB
    
    
    \end{document}
    