\documentclass[../../main/thesis_msc.tex]{subfiles}


\begin{document}

\chapter{Helium Stars as Progenitors of Thermonuclear Supernovae}

\textbf{Savvas Chanlaridis}, John Antoniadis, G\"otz Gr\"afener, and Norbert Langer (2019)[in preparation]

\begin{center}
\textbf{\large Abstract}
\end{center}
		
Type Ia supernovae (SNe\,Ia) are luminous optical transients characterized by the absence of hydrogen and helium in their spectra. 
The majority of SNe\,Ia are thought to result from the thermonuclear disruption of white dwarfs, which is triggered by mass accretion in a binary system. 
However, both the details of the explosion mechanism and the exact nature of the progenitor systems remain a topic of debate. 
We have discovered a novel SN\,Ia progenitor channel, in which helium stars in the initial mass range $1.8-2.7 \rm M_{\odot}$ that have lost their helium-rich mantle, develop a near-Chandrasekhar mass core via shell-burning, and experience a thermonuclear runaway at low densities. The explosive ignition of oxygen is triggered by burning amounts of residual carbon in the core, or by compressional heating in a small number of cases which exhibit a hybrid-like structure. This mechanism does not require accretion from the binary companion and therefore may contribute significantly to the SN\,Ia rate in star-forming galaxies (i.e. at early delay times) and explain some of the diversity amongst SN\,Ia spectra.

\section{Introduction} \label{sec:introduction}
Helium stars (He stars) are the exposed cores of stars that have lost their envelopes due to binary interactions. For progenitor zero-age main sequence (ZAMS) stars in the initial mass range M$_i \approx  7 - 11$\,M$_{\odot}$ that depends on metallicity and mixing processes 
\citep[e.g.][]{Ritossa1996, Ritossa1999, GilPons2005, siess2006, Poelarends2008, Farmer:2015afs}, carbon will be ignited in a shell, following the depletion of helium supply in their cores. Responsible for this off-centre ignition is significant neutrino cooling that occurs deep in the stellar core, and shifts the location of maximum temperature (temperature inversion) to the aforementioned shell. This marks the transition of the star toward the super-asymptotic giant branch (SAGB). The heat generated from the burning shell creates a subsonic carbon-burning front (``C-flame'') that propagates inward. The physical properties of such a deflagration have been the subject of various studies \citep[e.g.][]{Timmes_1994, siess2006, Siess2009, Denissenkov:2013qaa, Farmer:2015afs} and show a strong dependency on the adopted initial parameters.

As this C-flame moves toward the centre, it will process the material of the partially degenerate carbon-oxygen (CO) core converting it into an oxygen-neon (ONe) core. The subsequent evolution that will determine the final fate of the star, depends on the interplay between the core mass growth rate and the mass loss rate. If shell burning allows the core to reach the critical mass value of $\sim 1.37$ M$_{\odot}$ \citep{Nomoto1984}, the central density becomes sufficiently high ($\rho_c \sim 10^{9.95}$ gr cm$^{-3}$) for electron captures on $^{24}$Mg and $^{20}$Ne nuclei to ensue, essentially reducing the pressure in the interior, and ultimately leading to the collapse of the core; this is referred to as \textit{electron capture supernova} (ECSN).
The end product of an electron capture induced collapse would be a low-mass neutron star formed from a dim supernova \citep[e.g.][and references therein]{Fischer2010} with relatively low explosion energy, which imparts only a small natal kick to the remnant \citep{Knigge2011, Jones_2013, Jones2016} compared to the iron core collapse (Fe-CCSN) channel.

On the other hand, if the central density does not reach the threshold for electron captures on the most abundant species, the star will shed its envelope before the core reaches the Chandrasekhar mass limit (M$_{\text{ch}}$), and end up as an ONe white dwarf (ONe WD). However, if the ONe WD is in a close binary system, it can evolve to a supernova following a similar path as the one described above. This can be realized with the transfer of mass from the companion star onto the surface of the WD, allowing it to grow near the Chandrasekhar mass and leave behind a neutron star, a scenario that is known as \textit{accretion-induced collapse} (AIC) \citep[e.g.][]{nomoto1991, Schwab:2015bma, Brooks2017a, Schwab:2018cnb}.

\subsection{Thermonuclear Supernovae} \label{sec:thermonuclearSNe}
Due to lack of a hydrogen envelope, the collapsing core of a He star would be observed as a type Ib/c supernova, depending on the degree of stripping as has been shown by \cite{Tauris2013, Tauris2015}. In the case of normal SAGB stars that retain a hydrogen-rich mantle, the spectral lines could resemble the ones found in types IIn-P supernovae \citep[see][for details]{Moriya2014}.

Nonetheless, one should keep in mind that by the time the effects of electron captures on nuclei cannot be neglected, the necessary pressure gradient against gravity is provided by a strongly degenerate electron gas. For degenerate matter, pressure does not depend on the temperature thus the star can no longer respond to a temperature increase by expanding its outer layers. Since thermonuclear reaction rates are extremely sensitive to temperature variations, under such degenerate conditions, even a small increase of temperature caused by ignition of the existing nuclear fuel, could alter the nuclear burning rate dramatically resulting in a thermal runaway.

Whether the ONe WD undergoes core collapse or experiences a thermonuclear explosion depends on the timescale of electron captures versus the timescale of nuclear energy release. Since electron captures become energetically favorable only above specific values of central density, the ignition location of explosive nuclear burning plays a major role to the final result. Indeed, \cite{Jones2016, Jones2019} performed hydrodynamical simulations and were able to demonstrate that for low ignition density, the core does not collapse into a neutron star but rather explodes leaving behind bound remnants.

The consideration above is consistent with the explosion mechanisms (e.g. single degenerate model) related to type Ia supernovae \citep[for recent reviews see][]{Hillebrandt2000, Wang2012, Wang2018, Livio2018}. It becomes apparent that ECSNe could originate from a variety of progenitor systems either in single or binary configuration, and the connection between the progenitor and the final outcome is far from trivial.


\subsection{The Urca process}
The term ``Urca-processes" was introduced by \cite{Gamow1941} in order to describe energy losses from neutrino emission. It consists of two weak nuclear reactions: an electron capture that operates on a mother nucleus $M \equiv (A,Z)$ forming a more neutron-rich, daughter isobar $D \equiv (A,Z-1)$

\begin{align}
    \label{eq:EC}
    (A,Z) + e^{-} \longrightarrow (A,Z-1) + \nu_{e}
\end{align}

\noindent and a beta-decay transition of $D$ back to $M$

\begin{align}
    \label{eq:beta}
    (A,Z-1) \longrightarrow (A,Z) + e^{-} + \bar{\nu}_e
\end{align}

\noindent where $A$ and $Z$ denote respectively the mass number and the atomic number of the nucleus. The mother-daughter pair $(^A _Z{M}, ^A _{Z-1}{D})$ can be referred to as ``Urca nuclei". 

As central density increases, the Fermi energy $\epsilon_F$ (or equivalently the electron chemical potential, $\mu_e$) of the relativistic, degenerate Fermi gas prevails over the threshold energy $E_t$ of a given Urca nuclei pair (i.e. the difference between the rest masses of $M$ and $D$, given by the $Q$ value for a ground-state to ground-state transition), and electron captures will commence promptly. The associated reaction rates ($\lambda^{+}, \lambda^{-}$), and neutrino energy losses ($L^{+}, L^{-}$) per nucleon for equations \ref{eq:EC}, \ref{eq:beta} respectively, have been calculated by \cite{Tsuruta1970} and exhibit a strong sensitivity on temperature and density. Therefore, Urca processes become important only for a narrow range of stellar plasma properties, defining a thin Urca shell in which they operate (where $\epsilon_F = E_t$).

\cite{Paczy1972} argued that during the simmering phase of a CO WD, the energy released from carbon burning would not be able to be transferred efficiently via radiative means leading to a convective core that could engulf the Urca shell (convective Urca). Ultimately, the Urca neutrinos\footnote{Here we use the term ``neutrinos" to refer both to neutrinos and anti-neutrinos interchangeably.} would carry away enough energy to delay the dynamical runaway and forcing the core to move to higher densities thus, collapsing into a neutron star. However, \cite{Bruenn1973} challenged this notion by showing that Urca processes can also have destabilizing effects by generating heat, if convective motions re-position the relevant nuclei at some distance from the Urca shell. In these non-equilibrative states, if e-capture dominates over the $\beta$-decay (i.e. if $\rho > \rho_{\text{th}}$), the captured electron creates a ``hole" in the Fermi sea and forces another electron to drop from the Fermi surface in order to fill the gap, resulting in heating. On the other hand, if $\rho < \rho_{\text{th}}$, $\beta$-decay liberates electrons with excess thermal energy that also results in heating. Therefore, convective Urca processes can either play a major role as local cooling mechanisms (at mass coordinate in the vicinity of the Urca shell where both reactions are in equilibrium) or contribute to heating outside the Urca shell.

The effect of convective Urca processes on stellar interiors remain still an open question and provides up to this day fertile ground for debate. For a more detailed analysis, and fruitful discussion on the physics and importance of Urca process we refer to the work of \cite{Paczy1973, Barkat1990, Ritossa1999, Stein1999, Lesaffre2005, Waldman2007, Denisseknkov2015, Schwab2017}.

The structure of this paper is organized as follows. In Section\, \ref{sec:methods}, we present the input physics we used to model the evolution of our He-stars. In Section\, \ref{sec:results}, we discuss our results and their implications to our current state of knowledge. Section\, \ref{sec:discussion} provides a summary of our results and necessary future work.


% ---------------------------------------------------------------------------------
%                                   METHODS
% ---------------------------------------------------------------------------------

\section{Stellar Evolution Code and Initial Parameter Space} \label{sec:methods}
We performed numerical calculations using the one dimensional, stellar evolution code \textbf{M}odules for \textbf{E}xperiments in \textbf{S}tellar \textbf{A}strophysics (\mesa), version - r10398 \citep{Paxton2011, Paxton2013, Paxton2015, Paxton2018, Paxton2019}. Our \mesa inlists will become publicly available on \url{http://cococubed.asu.edu/mesa_market/inlists.html}.


\subsection{Physical assumptions} \label{sec:input_physics}
Our grid consists of 252 single He stars in the mass range $0.8 \leq M/M_{\odot} \leq 3.5$ with a step of $0.1$. Evolutionary calculations begin with our models being chemically homogeneous. In order to study the effects of metallicity, we create a series of models with low ($Z=0.0001$), intermediate ($Z=0.001$), and solar metallicity ($Z \equiv Z_{\odot} = 0.02$), where solar abundances are taken from \cite{grevesse1998}. Similarly, for varying the efficiency of overshooting we create series with no overshooting ($f_{\text{ov}} = 0.0$), and overshoot mixing ($f_{\text{ov}} = 0.014$, $f_{\text{ov}} = 0.016$) across all convective boundaries. The free parameter $f_{\text{ov}}$ is defined in \cite{Herwig2000} as a fraction of the local pressure scale height, $H_p$, and is connected to the diffusion coefficient, $D_{\text{OV}}$, via the relation given by equation \ref{eq:OV}, where $z$ is the geometric distance from the edge of the convective zone.
    					
    \begin{align}
    	\label{eq:OV}
    	D_{\text{OV}} = D_0 \exp \left( - \frac{2z}{H_v} \right), \hspace{0.5cm} H_v = f_{\text{ov}} \cdot H_p
    \end{align}

The choice of an adequate nuclear network and accurate weak rates are aspects of paramount importance in our considered mass-range, since we expect substantial amounts of $^{23}$Na, $^{25}$Mg, and $^{27}$Al to be produced after the carbon burning phase. Urca process operating on those odd-mass-number nuclei can alter the energy balance with a significant impact upon the thermal structure of the core. For this reason, we construct a nuclear reactions network that consists of forty-three nuclear species, including important NeNa and MgAl cycles, and relevant weak reactions for several Urca pair isotopes. Finally, we incorporate the special weak rates from \cite{Suzuki2016}.

We considered ion and electron screening corrections as described in \cite{PCR2009} and \cite{Itoh2002} respectively. To account for an enhanced carbon-oxygen mixture as a result of helium burning, we used the Type-2 OPAL Rosseland mean opacity tables \citep{OPAL}. For the equation of state blending we followed the options suggested by \cite{Schwab2017}.

Convection was treated according to the standard mixing-length theory prescription of \cite{MLT_Henyey} with a mixing-length parameter of $\alpha_{\text{ML}} = 2.0$. Furthermore, we used the Ledoux criterion for convective instability adopting an efficiency parameter of $\alpha_{\text{SEM}} = 1.0$ for semi-convection \citep{Langer1991}, and a diffusion coefficient $D_{\text{TH}} = 1.0$ for thermohaline mixing \citep{Brown_2013}. Both semi-convection and thermohaline mixing are treated by \mesa as diffusive processes \citep{Langer1983, Kipp_thermohaline}. 


Mass loss rates due to stellar winds were implemented using the ``\texttt{Dutch}" wind scheme \citep{Dutch}. In our case, this implies two different rates depending on the effective temperature of the star; for $T_{\text{eff}} > 10^4$ K and a surface abundance of hydrogen $X < 0.4$ by mass fraction (which is always satisfied for the models we developed), we apply the prescription of \cite{Nugis2000} with a scaling factor of $\eta = 1$ (canonical value). For $T_{\text{eff}} < 10^4$ K the mass loss rate follows the prescription of \cite{deJager1988}. 

All the baseline parameters we adopted above are summarized in Table \ref{tab:parameters}.

\begin{table}[t]

    \caption{Baseline parameters for single helium stars}
    \label{tab:parameters}
    \centering
        \begin{tabular*}{\linewidth}{@{\extracolsep{0.2\textwidth}}p{0.3\linewidth}p{0.3\linewidth}@{}}
        \hline \hline 
        Parameter & Value(s) \\
        \hline 
        Convection ($\alpha_{\text{\tiny{ML}}}$) & $2.0$ \\
        Semiconvection ($\alpha_{\text{\tiny{SC}}}$) & $1.0$ \\
        Thermohaline ($D_{\text{\tiny{TH}}}$) & $1.0$ \\
        Wind scaling factor ($\eta$) & $1.0$ \\
        Overshooting ($f_{\text{ov}}$) & $0.0$, $0.014$, $0.016$ \\
        Metallicity ($Z$) & $10^{-4}$, $10^{-3}$, $0.02$ \\
        \hline
        \end{tabular*}
\end{table}


\end{document}