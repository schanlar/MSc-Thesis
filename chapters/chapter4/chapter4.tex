\documentclass[../../main/thesis_msc.tex]{subfiles}


\begin{document}

	\chapter{Conclusions and Outlook}
	
		The observed plethora of SNe Ia signatures has created many classification subclasses, which in turn may have originate from distinct channels that have been proposed over the years. Currently, all these channels require a binary system although the explosion mechanism remains annoyingly nebulous. 


		In this thesis, we suggested a novel Type Ia supernova progenitor channel, that does not require a binary configuration in order to trigger an explosion. Within this paradigm, a low-mass helium star (a natural choice for a SN Ia candidate, since their spectra lack hydrogen lines) develops a degenerate ONe core after an off-centre carbon ignition. Rigorous shell burning forces the core to grow to near-Chandrasekhar mass whilst simultaneously the envelope is ejected via a strong wind. Our models are able to explosively ignite oxygen during the proto-WD stage, and avoid the core collapse triggered by electron captures. The culprit turns out to be amounts of unburnt carbon that have remained in the core during the earlier propagation of the convectively bound carbon-burning front.
		Moreover, the carbon flame fails to reach the centre in some of our models creating a hybrid-like structure. Compressional heating may ignite the carbon-rich core in a process that closely resembles typical SNe Ia progenitor channels.
		
		
		In section 3.1 we discussed a first-order approximation of the energetics, and the expected birth rates of such systems. We found that if these systems occur in nature indeed, they can account for a considerable fraction of the observed SNe Ia rate.
		
		
		% Future work focusing on distinct nucleosynthetic signatures from these systems,
		% binary configuration, include rotation
		 
		% Generic future work on treatments of convection, winds and mass loss
	
		




\end{document}