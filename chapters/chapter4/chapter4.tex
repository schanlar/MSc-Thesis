\documentclass[../../main/thesis_msc.tex]{subfiles}


\begin{document}

	\chapter{Conclusions and Outlook}
	
		The observed plethora of SNe Ia signatures has created many classification subclasses, which in turn may have originate from distinct channels that have been proposed over the years. Currently, all these channels require a binary system although the explosion mechanism remains annoyingly nebulous. 


		In this thesis, we suggested a novel Type Ia supernova progenitor channel, that does not require a binary configuration in order to trigger an explosion. Within this paradigm, a low-mass helium star (a natural choice for a SN Ia candidate, since their spectra lack hydrogen lines) develops a degenerate ONe core after an off-centre carbon ignition. Vigorous shell burning forces the core to grow to near-Chandrasekhar mass whilst simultaneously the envelope is ejected via a strong wind. Our models are able to explosively ignite oxygen at low densities during the proto-WD stage, and avoid the core collapse triggered by electron captures. The culprit turns out to be amounts of unburnt carbon that have remained in the core during the earlier propagation of the convectively bound carbon-burning front.
		Moreover, the carbon flame fails to reach the centre in some of our models creating a hybrid-like structure. Compressional heating may ignite the carbon-rich core in a process that closely resembles typical SNe Ia progenitor channels.
		
		
		In section 3.1 we discussed a first-order approximation of the energetics, nucleosynthetic signature, and the expected birth rates of such systems. We found that the available nuclear energy suffices to unbind the star and to produce ejecta with kinetic energies of the order $\sim 10^{51}$ erg, as we would expect from typical SNe Ia. Based on these results we argued that if these systems occur in nature indeed, they can account for a considerable fraction of the observed SNe Ia rate.
		
		In section 3.2 we performed a more detailed investigation by exploring a stellar model grid of 252 helium stars. Depending on the initial metallicity and overshoot mixing, we constrained the initial mass range to $\rm 1.8 - 2.7 \ M_{\odot}$.
		
		
		Obviously, our simulated results are greatly affected by the physics consideration in numerical models. Uncertainties related to stellar winds and mass loss at early stages, accurate weak rates for relevant nuclear reactions, as well as the treatment of convective mixing are only examples of some major issues that trouble the community of stellar astrophysics. Improvements on those processes would be extremely beneficial for modeling several aspects during the lifetime of a star.
		
		
		Focusing more on improving our results, although 1D stellar models consist a great tool, we need to follow-up our models with advanced 3D hydrodynamical simulations in order to produce synthetic spectra, predict complex nucleosynthesis, and expected explosion energies.
		Additionally, true binary evolution of our models could provide an insight on mass-loss histories, and improve constraints on their progenitor systems. Finally, rotation and the influence of magnetic fields are both aspects that are worth investigating, since they can alter the overall evolution significantly.
		
		




\end{document}