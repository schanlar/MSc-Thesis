\documentclass[../../main/thesis_msc.tex]{subfiles}

\begin{document}

\chapter{Conclusions and Outlook}\label{ch:conclusions}
	
		The observed plethora of SNe Ia signatures has created many classification subclasses, which in turn may have originate from distinct channels that have been proposed over the years. Currently, all these channels require a binary system although the explosion mechanism remains annoyingly nebulous. 


		In this thesis, we suggested a novel Type\,Ia supernova progenitor channel, that does not require a binary configuration in order to trigger an explosion. Within this paradigm, a low-mass helium star (a natural choice for a SN\,Ia candidate, since their spectra lack hydrogen lines) develops a degenerate ONe core after an off-centre carbon ignition. Vigorous shell burning allows the core to grow to near-Chandrasekhar mass whilst simultaneously the envelope is ejected via a strong wind. Our models are able to explosively ignite oxygen at low densities during the proto-WD stage, and avoid the core collapse triggered by electron captures. The culprit turns out to be amounts of unburnt carbon that have remained in the core during the earlier propagation of the convectively bound carbon-burning front.
		Moreover, the carbon flame fails to reach the centre in some of our models creating a hybrid-like structure. Compressional heating may ignite the carbon-rich core in a process that closely resembles typical SNe Ia progenitor channels.
		
		
		In Chapter\,\ref{ch:paperI} we discussed a first-order approximation of the energetics, nucleosynthetic signature, and the expected birth rates of such systems. We found that the available nuclear energy suffices to unbind the star and to produce ejecta with kinetic energies of the order $\sim 10^{51}$ erg, as we would expect from typical SNe Ia. Based on these results we argued that if these systems occur in nature indeed, they can account for a considerable fraction of the observed SNe Ia rate.
		
		In Chapter\,\ref{ch:paperII} we performed a more detailed investigation by exploring a stellar model grid of 252 helium stars. Depending on the initial metallicity and overshoot mixing, we constrained the initial mass range of potential candidates to $\rm 1.8 - 2.7\, M_{\odot}$.
		
		
		%Obviously, our simulated results are greatly affected by the physics consideration in numerical models. Uncertainties related to stellar winds and mass loss at early stages, accurate weak rates for relevant nuclear reactions, as well as the treatment of convective mixing are only examples of some major issues that trouble the community of stellar astrophysics. Improvements on those processes would be extremely beneficial for modeling several aspects during the lifetime of a star.
		
		
		Although 1D stellar evolution modeling consists a great tool, there are several directions that we could take in order to improve upon our models.
		\begin{itemize}
			\item One of the main sources of uncertainty is the accuracy of numerical calculations during the late stages of the evolution. It is unclear if our simulations can accurately reproduce the structure and properties of the outermost layers of stars with convective envelopes. For this reason, more simulations are needed in order to understand the complex underlying physics of stellar winds and envelope ejection.
			
			\item In addition, other relevant mechanisms for envelope ejection can arise as a result of several binary interactions. Thus, binary evolution calculations would give us the opportunity to realistically probe various formation channels for our models, and to accurately map the mass-loss history of the binary system. Moreover, such models would enable us to investigate the influence of other parameters such as rotation and magnetic fields.
			
			\item  The aforementioned binary evolution calculations can provide us with a more accurate parameter space regarding the explosibility of such stars. Based on this parameter space, one could perform population synthesis calculations in order to investigate the arising delay time distribution from such systems, and its connection to our current picture of the delay time distribution of SN\,Ia.
			
			\item Advanced 3D hydrodynamical simulations of our progenitor models are required, in order to explore the geometry and the energetics of a potential explosion. Additionally, calculations on time-dependent radiative transfer would give us the opportunity to calculate synthetic spectra and expected light curves from this SN\,Ia channel.
			
			\item Accurately predicting nucleosynthetic yields is yet another important aspect that requires 3D modeling. Preliminary results suggest that this channel can give rise to distinct nucleosynthetic signatures that should be traceable in spectra of stars [ref].
			
			\item Finally, by comparing the aforementioned nucleosynthesis yields with solar and stellar abundances, we can establish stringent observational constraints for this SN\,Ia channel.
			
		
		\end{itemize}
		

		

\end{document}
