\documentclass[../main/thesis_msc.tex]{subfiles}


\begin{document}

\chapter*{Abstract}

Thermonuclear explosions of carbon-oxygen white dwarfs are being observed as Type Ia supernovae. 
%Within the standard paradigm, mass accretion onto the carbon-oxygen white dwarf, either during a stable mass transfer phase, or in a merger, triggers  runaway nuclear burning.
Although the scientific consensus is that  these transients originate in binaries, their progenitors as well as the associated explosion mechanism are much debated. 

SNe Ia exhibit significant diversity in terms of their spectra and light curves, and thus, it is rather unclear if the companion star to the white dwarf is a non-degenerate star (single degenerate scenario), or another white dwarf (double degenerate scenario). In addition, there may be significant contributions from alternative channels. 

In this thesis, we performed detailed numerical calculations with the \mesa stellar evolution code in order to examine models of single, non-rotating helium stars as potential progenitors of SNe Ia. By adopting various initial values for the metallicity, and by allowing efficient overshooting mixing across convective boundaries, we found that the vast majority of He-stars in the initial mass range $\rm 1.9-2.7\, M_{\odot}$ develop a degenerate ONe core, that retains a small amount of carbon. Subsequent shell burning can increase the mass of the core to the Chandrasekhar mass limit and ignite the residual carbon, resulting in a thermonuclear explosion. 

In some limiting cases, the carbon-burning front that was created after helium depletion in the core, is stalled before it reaches the centre. This is caused by mixing that reduces the efficiency of carbon-burning, resulting in a hybrid-like structure where the inner CO core is surrounded by a processed ONe mantle. We found that even in this case, if the core of the hybrid star reaches the Chandrasekhar mass limit, compressional heating will ignite the carbon-rich core in a similar fashion to typical SNe\,Ia progenitors.

By performing a first-order estimation of the birth rate of these systems, we argue that in the absence of an extended accretion phase, these models can provide an alternative channel for SNe\,Ia progenitor systems with a non-negligible effect on the observed rate of SNe Ia.


\end{document}