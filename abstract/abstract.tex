\documentclass[../main/thesis_msc.tex]{subfiles}


\begin{document}

\chapter*{Abstract}

Thermonuclear explosions of white dwarfs are being observed as Supernovae-Ia. The progenitors, and the exploding mechanism of such phenomena are topics with a lot of room for debate, although the scientific consensus is that these systems originate from binaries in which mass transfer, or coallesence of the binary triggers a runaway process in the deep interior of the white dwarfs leading to a catastrophic explosion without any remnants.

Since SNe-Ia exhibit significant variations in terms of their spectra, is is rather unclear if the companion star is a non-degenerate star (single degenerate scenario, SD), or another white dwarf (double degenerate scenario, DD), or even some combination of scenarios. In this thesis, we performed numerical caluclations exploiting the MESA stellar evolution code in order to examine single, non-rotating Helium stars as potential progenitors of SNe-Ia. By adopting various values for the metallicity environment, and by allowing efficient overshooting mixing, we found that He-stars in the mass range $1.8-2.7$ M$_{\odot}$ develop a degenerate ONe core, that retains a small amount of carbon. Subsequent shell burning can force the degenerate ONe core to grow to near-Chandrasekhar mass, igniting the residual carbon and destabilizing the star, ultimately resulting in a thermonuclear explosion.

In the absence of an extended accretion phase, these models can provide an alternative channel within the SD scenario, which can have a non-negligible effect on the observational rates of SNe-Ia.


\end{document}